\documentclass{article}

\usepackage{fullpage}
\usepackage{graphicx}
\usepackage{subfig}

\newcommand{\tbbmap}{\texttt{concurrent\textunderscore hash\textunderscore map}}
\newcommand{\libcuckoo}{\texttt{libcuckoo}}

\title{Comparing Intel TBB {\tbbmap} and \libcuckoo}
\author{Manu Goyal, Dave Andersen, Michael Kaminsky}
\date{\today}

\begin{document}

\maketitle{}

\section*{Overview}

In this benchmark, we compare {\libcuckoo}, our high-performance,
memory-efficient hash table, with the Intel Thread Building Blocks {\tbbmap}. We
compare the performance of the two tables across different types of workloads
and different numbers of cores, and also compare the memory usage for different
table sizes. We ran the benchmarks on a machine with 36 Intel Xenon 2.9 GHz
cores and 60GB memory. The cores were split evenly into two NUMA nodes.

\section*{Pure Read}
Our read benchmark fills a table up to 90\% of its allocated capacity, then
concurrently runs reads for data that is in the table and data that isn't. It
counts the number of reads executed over 10 seconds. Figure~\ref{fig:pure-read}
compares the read throughput of the two tables with integer and string keys. For
both types of keys, {\libcuckoo} outperforms {\tbbmap}, with the difference
getting slightly larger as we increase the number of threads. With 32 threads,
{\libcuckoo} outperforms {\tbbmap} by 47\% for integers, and 37\% for string
keys.

\begin{figure}[!htbp]
  \centering
  \subfloat[]{{\includegraphics[width=0.5\textwidth]{read-integer}}}
  \subfloat[]{{\includegraphics[width=0.5\textwidth]{read-string}}}
  \caption{Pure read throughput for integer and string keys}
  \label{fig:pure-read}
\end{figure}

\section*{Pure Insert}
Our insert benchmark measures the time taken to fill up a table from 0\% to 90\%
of its allocated capacity. Figure~\ref{fig:pure-insert} compares the insert
throughput with integer and string keys. For integers, {\libcuckoo} greatly
outperforms {\tbbmap}, by over 680\%, and for strings, it outperforms {\tbbmap}
by 50\%. We suspect that TBB's low performance on integers was due to the fact
that it doesn't deal well with multiple NUMA clusters. This would explain the
poor scaling with large numbers of threads on different NUMA clusters. With
strings, since the cost of copying strings likely dominates the runtime, this
effect is less apparent.

\begin{figure}[!htbp]
  \centering
  \subfloat[]{{\includegraphics[width=0.5\textwidth]{insert-integer}}}
  \subfloat[]{{\includegraphics[width=0.5\textwidth]{insert-string}}}
  \caption{Pure insert throughput for integer and string keys}
  \label{fig:pure-insert}
\end{figure}

\section*{Mixed Workload}
Our mixed benchmark runs a mixed workload of inserts and reads at a configurable
ratio, and measures the time and number of operations taken to fill up the table
from 0\% to 90\% of its allocated capacity. Figure~\ref{fig:mixed-read-insert}
compares the performance of the two tables at different ratios of inserts (all
with 32 threads), with {\libcuckoo} doing better with both integer and string
keys. We see again that the difference between the two tables is much greater at
higher insert percentages compared to lower percentages (588\% compared to 36\%,
respectively), because the difference between {\libcuckoo} and {\tbbmap} is more
pronounced for inserts than it is for reads.

\begin{figure}[!htbp]
  \centering
  \subfloat[]{{\includegraphics[width=0.5\textwidth]{read-insert-integer}}}
  \subfloat[]{{\includegraphics[width=0.5\textwidth]{read-insert-string}}}
  \caption{Mixed read-insert throughput for integer and string keys}
  \label{fig:mixed-read-insert}
\end{figure}

\section*{Memory Usage}
Finally, we compare the approximate memory usage of the two tables. While not a
completely accurate measure of the amount of memory used by each table, we
measured the maximum resident set size as determined by Ubuntu's \texttt{time}
command for the insert benchmark. For integers, {\libcuckoo} scales far better
than {\tbbmap}, using 72\% less memory than {\tbbmap} with the largest table.

\begin{figure}[!htbp]
  \centering
  \includegraphics[width=0.5\textwidth]{mem}
  \caption{Memory usage for insert benchmark with integer keys. Table size is
    the number of elements each table has capacity for}
  \label{fig:mem}
\end{figure}

\section*{Conclusion}

{\libcuckoo} has a number of features that cause it to perform better and use
less memory than {\tbbmap}. {\libcuckoo} stores data in a cache-optimized form
and avoids false sharing between CPU's, which allow it to scale inserts and
reads very well to a large number of CPU's with low memory overhead.
Furthermore, the cuckoo hashing algorithm lets it achieve very high table load
factors before needing to expand, which significantly reduces memory usage.

\end{document}

%%% Local Variables:
%%% mode: latex
%%% TeX-master: t
%%% End:
